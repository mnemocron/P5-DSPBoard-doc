\subsubsection{Anforderungen an Microcontroller}
\label{sec:Konzept_Microcontroller}

Die an den Prozessor gestellten Anforderungen sind ein ARM-Cortex M4 Core mit und FPU sowie Schnittstelle(n) zur Kommunikation mit dem Audio Codec. 
Dabei wird aufgrund der genaueren Samplingrate der Codec als Master betrieben und der DSP als Slave.
Der Microcontroller muss also keine genaue Clock zur Verfügung stellen. 
Auf eine Cortex-M7 Architektur wird verzichtet, weil ab diesem Punkt auch für ein Single-Board Computer (vgl. Raspberry Pi) argumentiert werden kann.
Eine Tacktfrequenz von 200MHz ist wünschenswert, jedoch befinden sich die Cortex-M4 Prozessoren mit mehr als 100MHz auf dem selben Preisniveau von Cortex-M7 Microcontrollern.

\begin{table}[H]
	\centering
	\begin{tabular}{|c|c|c|c|c|c|}
		\hline
		\textbf{MCU}  & \textbf{Flash {[}kB{]}} & \textbf{RAM {[}kB{]}} & \textbf{Clock {[}MHz{]}} & \textbf{Preis} & \textbf{Core} \\ \hline
		STM32F407VGTx & 1024                    & 192                   & \textbf{168}             & \textbf{5.632} (10kU) & M4            \\ \hline
		\textbf{STM32F412RETx} & 512            & 256                   & 100                      & 3.435 (10kU)        & M4            \\ \hline
		STM32F722RCTx & 256                     & 256                   & 216                      & 3.663 (10kU)        & \textbf{M7}   \\ \hline
		ATSAMD51P20A  & 1024                    & 256                   & 120                      & 4.33 (5kU)          & M4            \\
		\hline
		S6E2CCAJ0A    & 2048                    & 256                   & \textbf{200}             & > 11.00             & M4            \\
		\hline
	\end{tabular}
 	\caption{Parametervergleich verschiedener Cortex-M Microcontrollern}
 	\label{tab:ComparisonMCU}
\end{table}

\textit{Quelle der Preise: STMicroelectronics, Microchip (Atmel), Octopart (Cypress)} \\

Verwendet wird basierend auf dem Preis und 


