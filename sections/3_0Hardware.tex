\section{Hardware}
\label{sec:Hardware}


In der Aufgabenstellung wird beschrieben, dass die Mechanik des 3D-Druckers K8200 von Velleman zur Verfügung gestellt wird. In Absprache mit dem Auftraggeber wurde entschieden, dass der 3D-Drucker Ender 3 Pro von Creality verwendet wird. Dieser hat den Vorteil ein grösseres Druckvolumen bei kleinerem Baugrösse anzubieten, weiter ist das Heizbett bis $110 \,\si{\degree C}$ aufheizbar und nicht nur  bis $60 \,\si{\degree C}$ was das Drucken von ABS ermöglicht. Weiter ist ein Display plus Bedienelement bereits vorhanden.\\
Im folgenden Unterkapitel wird beschrieben, welche mechanischen und elektrischen Komponenten vom Ender 3 Pro übernommen sowie auch, welche Änderungen vorgenommen wurden. Weiter wird sowohl das Schema mit dessen Komponenten und die Überlegungen dahinter als auch das Layout des PCBs erläutert.
