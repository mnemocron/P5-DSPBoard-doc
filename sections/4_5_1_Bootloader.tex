\subsubsection{Bootloader starten}
\label{sec:Enter_DFU}

Der STM32 muss in den internen Bootloader starten. Dies könnte theoretisch über Software (Assembler) per Anpassung der \texttt{startup.s} Datei erreicht werden.
Jedoch ist auf dem DSP Board eine Schaltung vorhanden, um den \texttt{BOOT0} Pin vorübergehend auf \texttt{HIGH} zu ziehen.
Wenn ein Starten des Bootloaders und somit des DFU Modus gewünscht wird, muss dies in der Software folgendermassen realisiert werden.
\\
\begin{lstlisting}[style=Cuvision,caption={Starten des Bootloaders durch CPU Reset}]
/* If User Button is pressed on Startup, enter DFU Firmware Upgrade Mode */
if(HAL_GPIO_ReadPin(SW2_GPIO_Port, SW2_Pin) == GPIO_PIN_SET){
  // pull BOOT0 = 1
  HAL_GPIO_WritePin(SET_BOOT0_GPIO_Port, SET_BOOT0_Pin, GPIO_PIN_SET);
  HAL_Delay(500);      // wait for Capacitor to charge to ~3.3V
  NVIC_SystemReset();  // Reset the MCU
}
\end{lstlisting}

Das oben aufgeführte Listing beschreibt das Auslösen des DFU Modus sobald der einer der User Buttons gedrückt wird.
Damit der Kondensator Zeit hat, sich aufzuladen, wird ein Delay von ca. 500ms benötigt.
Anschliessend wird mit dem Befehl \texttt{NVIC\_SystemReset()} ein Software Reset ausgelöst.
Der STM32 sollte nun in den Bootloader starten.

Vor dem Systemreset besteht die Möglichkeit, einen Hinweisstring (z.B. \textit{"upgrading..."}) auf dem OLED Display anzuzeigen.
Der Text bleibt während dem Upgrade bestehen, da die Spannungsversorgung nicht unterbrochen wird.



