\section{Einleitung}
\label{sec:Einleitung}

In den Bereichen Amateurfunk und Hobbymusik gibt es viele Situationen in denen ein einfaches, DSP-basiertes Effektgerät zur Anwendung gebracht werden kann. So soll beispielsweise ein Notchfilter einen Störton unterdrücken, oder auf Knopfdruck ein Reverb-Effekt eingeschaltet werden können. \\

Das derzeit verwendete DSP Board für den Unterricht im MicroCom Labor basiert auf einem dsPIC33 mit Fixed-Point-Recheneinheit. Die neuen ARM Prozessoren bieten ab der Cortex-M4 Serie eine Floating-Point-Unit (FPU) und ermöglichen dadurch eine schnellere Verarbeitung von Signalen. \\

Aus diesem Grund wird die Hardware des DSP Boards überarbeitet und soll mit einem ARM Cortex-M4 Microcontroller ausgestattet werden. Der Schaltungsaufwand beschränkt sich auf die wesentlichen Funktionen. Diese beinhalten die MCU, einen Codec für die AD/DA Wandlung, die Audio-Steckverbinder und die Bedienelemente des HMI. \\

Im Bereich Amateurfunk und Hobbymusik besteht oft ein Bedürfnis nach einer einfachen Möglichkeit, ein Audiosignal mit einem Effekt zu verändern.
So kann es sein, dass ein Amateurfunker mit einem Notch-Filter einen Störton unterdrücken möchte. Als Musiker möchte man mit einer Effektbox einen Reverbeffekt erzeugen.
Effektgeräte und Filter am Markt sind oft zu einem Premiumpreis erhältlich.
Dieses Projekt hat zum Ziel, eine günstige Alternative zu diesen Geräten zu bieten.


Heute bieten die DSP Funktionen in der ARM Cortex-M4 Architektur einegünstige Möglichkeit Signalverarbeitung auf Microcontrollerebene zu betreiben. 
Der Rahmen dieses Projektes umfasst die Entwicklung der Hard- und Firmware eines DSP Boards mit ARM Cortex-M4 Microcontroller. 
Das Gerät wird mit Bedienelementen wie 2 Dreh
