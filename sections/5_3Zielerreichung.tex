\subsection{Zielerreichung}
\label{sec:Zielerreichung}
\todo[inline]{Fertig - MR - Hier wird anhand der Resultate aus dem Kapitel «Funktionstest» gezeigt, inwiefern die Ziele aus dem Pflichtenheft erfüllt sind.  - A: MR}

Zur Ableichung des fertiggestellten Projekts mit dem Pflichtenheft, sind in diesem Kapitel die Soll- und Wunschziele aus dem Pflichtenheft mit Angabe zur Erreichung aufgeführt.

In der Tabelle \ref{tab:sollziele} sind die Sollziele aufgeführt, die zur erfolgreichen Fertigstellung des Projekts erreicht werden müssen. In der Tabelle \ref{tab:wunschziele} sind die wünschenswerten Ziele inklusiv Status aufgelistet. Wie aus den genannten Tabellen hervorgeht, konnten alle Soll- wie auch Wunschziele erreicht werden.

Im Folgenden sind einige Bemerkungen zu gewissen Zielen aufgeführt.\\

\paragraph{Bemerkungen zu gewissen Zielen} 
\vspace{-4mm}
\begin{itemize}
	\item Sollziel 1 \& 4: Der Drucker K8200 wurde in Absprache mit dem Auftraggeber durch den Drucker Ender 3 Pro ersetzt. Die Ziele blieben sinngemäss die selben.
	\item Wunschziel 1: Die Reduzierung der Sensoren konnte auf der X und Y Achse mittels StallGuard realisiert werden. Auf der Z Achse war dies aufgrund von Stromspitzen nicht möglich. Ein zusätzlicher Sensor für das Filament wurde hinzugefügt.
	\item Wunschziel 4: Der Lüfter am Extruder und derjenige am Print laufen nur beim Betrieb des Extruders resp. der Motorentreiber. Verglichen mit dem Ender 3 Pro, benötigt der Drucker aufgrund des ESP32 aber mehr Leistung. 
\end{itemize}


\begin{table}[!htbp]
	\small
	\def\arraystretch{1.1} \tabcolsep=14pt
	\begin{tabular}{|m{0.6cm}|m{2.5cm}|m{7.6cm}|C{1.3cm}|}
		
		\hline 
		\textbf{Nr.}
		&
		\textbf{Sollziele}
		&
		\textbf{Beschreibung}
		&
		\textbf{Status}
		\\ \hline
		1&
		Drucker
		& Die bestehende Steuerung des 3D-Druckers Velleman K8200 wird während dieses Projektes durch eine eigens entwickelte Steuerung ersetzt. Die Mechanik des Druckers wird vom Original übernommen. In Absprache mit dem Auftraggeber kann auch ein anderes Druckermodell verwendet werden.
		&
		\checkmark
		\\ \hline
		2&
		Interpretieren von G-Code                                                          & Der fertiggestellte Drucker wird in der Lage sein, 3D-Druckaufträge anhand von G-Code Files zu drucken.  
		&
		\checkmark
		\\ \hline
		3&
		Speisung                                                                           & Die Speisung erfolgt über ein externes Steckernetzteil oder Netzmodul.  
		&
		\checkmark
		\\ \hline
		4& 
		Aufbau                                                                             & Die komplette Steuerung wird in einem Print integriert und in die Mechanik des Druckers K8200 eingepasst. Am Extruderkopf kann ein kleiner Print angebracht werden, falls dadurch die Kabelführung optimiert wird. 
		&
		\checkmark
		\\ \hline
		5& 
		Drucken ohne externen PC                                                           & Drucken wird ohne PC möglich  sein, sofern die entsprechenden G-Code Daten vorhanden sind.                                                                                           
		&
		\checkmark
		\\ \hline
		6& 
		Wireless Drucken                                                                   & Eine Verbindung zum Drucker kann über eine Funkverbindung hergestellt werden. Über die Funkverbindung kann der Druck gestartet, sowie der Druckfortschritt überwacht werden. 
		&
		\checkmark
		\\ \hline
		7& 
		SD-Karte                                                                           & Eine SD-Karte dient als Datentransfer für Druckaufträge und als Speicher für die Steuerung. Ein kompletter Druckauftrag soll gespeichert werden können. 
		&
		\checkmark
		\\ \hline
		8& 
		Ansteuerung Motoren                                                                 & Die Steuerung muss alle Achsen (X, Y und Z), sowie den Filamentvorschub, schrittverlustfrei ansteuern können.   
		&
		\checkmark
		\\ \hline
		9& 
		Anzeige                                                                            & Am Drucker wird eine Anzeige angebracht, die über eine Eingabeeinheit bedient werden kann. 
		&
		\checkmark
		\\ \hline
	\end{tabular}
	\caption{Sollziele aus dem Pflichtenheft | \checkmark: erreicht  |  \text{\sffamily X}: nicht errreicht }
	\label{tab:sollziele}
\end{table}


\begin{table}[!htbp]
	\small
	\def\arraystretch{1.1} \tabcolsep=14pt
	\begin{tabular}{|m{0.6cm}|m{2.5cm}|m{7.6cm}|C{1.3cm}|}
		\hline 
		\textbf{Nr.}
		&
		\textbf{Wunschziele} 
		& 
		\textbf{Beschreibung}
		&
		Status
		\\ \hline
		1&
		Reduktion / Verbesserung Sensoren  &  Falls möglich soll die Anzahl Sensoren, Schalter und Endschalter, im Vergleich zum ursprünglichen Drucker, reduziert oder optimiert werden.  
		&
		\checkmark
		\\ \hline 
		2&
		Nachlauf  &  Eine Nachlauferkennung für das Filament soll, wenn möglich, implementiert werden. 
		&
		\checkmark
		\\ \hline 
		3&
		Druckfortschritt Anzeige  &  Der Druckfortschritt soll nach Möglichkeit auf der Anzeigeeinheit ersichtlich sein.  
		&
		\checkmark
		\\ \hline 
		4&
		Power Save  &  Der Drucker soll in einen Energiesparmodus gehen, wenn nicht gedruckt wird.  
		&
		\checkmark
		\\ \hline 
		5&
		Heizbett  &  Die Hard- und Software sollen Unterstützung für ein optionales Heizbett bieten.  
		&
		\checkmark
		\\ \hline                                                    
	\end{tabular}
	\caption{Wunschziele aus dem Pflichtenheft | \checkmark: erreicht  |  \text{\sffamily X}: nicht errreicht }
	\label{tab:wunschziele}
\end{table}