\paragraph{Gehäuse}
Das Gehäuse besteht aus zwei, beziehungsweise vier, Teilen - der Bodenplatte und dem Deckel. Da das Gehäuse selbst 3D-gedruckt ist, untersteht es auch deren Limitation, der eingeschränkten Druckfläche. Deshalb sind die beiden Teile jeweils wieder in zwei Hälften unterteilt. Die beiden Grundteile werden nachstehend beschrieben.

\begin{figure}[h]
	\centering
	\includegraphics[width=0.7\linewidth]{case}
	\caption{Diese Grafik zeigt den Aufbau des Gehäuses. Der rote Teil ist der Deckel und der blaue Teil ist die Bodenplatte.}
	\label{pic:Case}
\end{figure}


\paragraph{Bodenplatte}
Die Bodenplatte ist in Abbildung \ref{pic:Case} in blau dargestellt und ihre Hauptaufgabe liegt darin, den Print zu halten. Ausserdem ist die Frontplatte auch Teil der Bodenplatte. Beim Design wurde vor allem auf die Positionierung des Prints geachtet, sodass der Print unter dem Rahmen und dem Druckbett Platz hat. Des Weiteren wurde die Orientierung des Prints so gewählt, dass die USB-Buchse und der SD-Karten Slot gut erreichbar sind. Weiter ist beim SD-Karten Slot eine tiefe Fase eingefügt, damit die SD-Karte einfacher erreichbar ist. An der Bodenplatte sind Lüftungsschlitze vorhanden, damit der Print ausreichend gekühlt wird.

\paragraph{Deckel}
Der Deckel (in Abbildung \ref{pic:Case} in rot) dient in erster Linie als Berührungsschutz. Der Deckel ist am Rahmen direkt mit vier Schrauben angeschraubt und liegt am Frontblech der Bodenplatte auf. Da der Deckel auf gleicher Höhe ist, wie die Y-Achsaufhängung, ist für diese auch noch eine Aussparung vorgesehen. Weiter wird ein Lüfter zum Kühlen des PCBs am Deckel montiert.

\todo[inline]{Evtl. Bild anpassen (mit Schlitzen für Lüfter)\\OL - Lüftermontage erwähnen. - A: MR}
