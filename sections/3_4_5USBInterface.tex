\subsubsection{USB / UART Schnittstelle}
\label{sec:SchemaUSBInterface}

Sowohl die Firmware Marlin wie auch die Firmware ESP3D verfügen über Optionen, welche eine Kommunikation über UART erfordern. Marlin kann optional über eine UART Schnittstelle gesteuert werden. Dazu wird zulässiger G-Code über die serielle Schnittstelle übertragen, welcher dann von der Firmware Marlin interpretiert und ausgeführt wird. Die Firmware ESP3D benötigt die UART Schnittstelle für eine eventuelle Ausgabe von Debug-Informationen. Zusätzlich wird der ESP32 über die serielle Schnittstelle programmiert. Um den einfachen Anschluss an moderne Computer zu ermöglichen ist ein USB-zu-UART-Wandler verbaut.

Dabei handelt es sich um einen FT2232D Chip vom Hersteller FTDI. Dieser bietet ähnliche Funktionen wie der FT232 Chip an, verfügt allerdings über zwei UART Schnittstellen. Diese erscheinen an einem Computer als zwei separate virtuelle COM-Ports. Dadurch ist es möglich, sowohl die UART Schnittstelle des STM32 wie auch die UART Schnittstelle des ESP32, mit nur einem Chip und einer USB Schnittstelle mit einem Computer zu verbinden.

Die Beschaltung des FT2232D orientiert sich dabei an mehreren Beispielkonfigurationen aus dem Datenblatt und ist in Abbildung \ref{pic:Schema_USB} ersichtlich. Die Speisung erfolgt von der $5\,\si{\volt}$ Versorgung, da der FT2232D über einen internen $3.3\,\si{\volt}$ Wandler verfügt. Zusätzlich wird noch eine $3.3\,\si{\volt}$ Spannungsversorgung für die IO-Pins angeschlossen.
Dies hat den Grund, dass der im FT2232D interne $3.3\,\si{\volt}$ Wandler nur einen Strom von $<5\,\si{mA}$ liefern kann, was für den Betrieb vom FT2232D Chip und den zusätzlichen 4 LEDs nicht ausreicht.

Nach Empfehlung von FTDI sollten die $3.3\,\si{\volt}$\ IO Spannungsversorgung und die $5\,\si{\volt}$\ Versorgung denselben Ursprung haben.
Daher, wenn die $3.3\,\si{\volt}$\ IO Spannungsversorgung vom Board bereitgestellt wird, sollte auch die $5\,\si{\volt}$\ Versorgung vom Board bereitgestellt werden. Aufgrund dieser Anforderung wird die $5\,\si{\volt}$\ Speisung von der USB Schnittstelle (VBUS) nicht zur Spannungsversorgung verwendet.
Diese wird nur zum resetten des FT2232D Chips benutzt.

\begin{figure}[h]
	\centering
	\includegraphics[width=0.9\linewidth]{USB_Schema.pdf}
	\caption{Schema der USB zu UART Schnittstelle}
	\label{pic:Schema_USB}
\end{figure}

Entkopplung sowie Filterung der analogen Speisung (AVCC) erfolgen ebenfalls nach den Empfehlungen von Hersteller FTDI \cite{FT2232D}.

Die vier LEDs zeigen die Aktivität auf der \texttt{Tx} und der \texttt{Rx} Leitung der entsprechenden Schnittstelle an. Das externe EEPROM zur Konfiguration des FT2232D Chips wird nicht benötigt, da die Standardeinstellungen vom FT2232D Chip den benötigten Einstellungen (zwei UART Schnittstellen) entsprechen.
