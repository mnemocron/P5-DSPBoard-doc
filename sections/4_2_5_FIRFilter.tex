\subsubsection{FIR Filter}
\label{sec:LibFIRFilter}

Als Wrapper für die CMSIS/DSP Funktionen für FIR Filter dienen die Dateien \texttt{fir.h} und \texttt{fir.c}. In dieser Library sind die Filterkoeffizienten in einem statischen Array abgelegt.


\paragraph{FIR Filter initialisieren}

Zusätzlich sind Init-Funktionen für das Mono- und Stereo-FIR Filter vorhanden. 
Das Initialisieren durch die Init Funktion muss vor dem Aufrufen der FIR-Processing Funktion geschehen, da die MCU sonst in einen Hard Fault läuft.

Im \texttt{main.c} muss der nachfolgende Code zur Initialisierung aufgeruffen werden.

\begin{lstlisting}[style=Cuvision, caption={Init Funktion der FIR Filter}]
/* USER CODE BEGIN 2 */
FIR_Init_Mono();
FIR_Init_Stereo();
/* USER CODE END 2 */
\end{lstlisting}


\paragraph{Filterkoeffizienten mit MATLAB berechnen}

Die Berechnung eines konkreten Filters ist nicht Teil dieses Projekts.
Für die Beispielimplementation des FIR Filters wurde für das Mono-Filter das Tiefpassfilter Example aus der CMSIS/DSP Dokumentation verwendet. 
Eine Kopie des Beispielcodes ist im Projekt unter \\
\texttt{./Drivers/CMSIS/DSP/Examples/ARM/arm\_fir\_example/arm\_fir\_example\_f32.c} enthalten.
Die Dokumentation des Filters bzw. der gesamten CMSIS/DSP Library ist auf der Webseite von ARM \cite{cmsis-doc-arm} oder Keil \cite{cmsis-doc-keil} abrufbar.

Für das Stereofilter wurde mit dem \texttt{fir1} MATLAB-Befehl die 11 Filterkoeffizienten berechnet. Das Filter hat eine 3dB-Grenzfrequenz bei $6/24 \pi rad/Sample$.

Die MATLAB-Funktion \texttt{fir1} arbeitet mit der \textit{Windowing-Methode} \cite{FIR-Windowing}, die nächer im Abschnitt \ref{sec:LibFIRAdaptive} beschrieben wird.
Dabei steht das erste Argument (hier: 10) für den Filtergrad oder $Anzahl Koeffizienten\ -\ 1$. Der zweite Parameter ist die auf die Samplingfrequenz normierte Grenzfrequenz.

\begin{lstlisting}[language=matlab]
fir1(10, 6/24)
\end{lstlisting}


