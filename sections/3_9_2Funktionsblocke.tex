\subsubsection{Funktionsgruppen}
\label{sec:Funktionsblocke}

Die Funktionsgruppen des PCB-Layouts entsprechen grundsätzlich der Unterteilung des Schemas.
Zusätzlich werden analoge und leistungsführende Schaltungsteile separat behandelt.
Die Abbildung \ref{pic:Layout_Block} zeigt die Anordnung der Funktionsgruppen auf dem PCB.
Die linke untere Hälfte des PCBs füllen die Schrittmotortreiber.
Daneben befindet sich der SD-Karten Slot und die USB Typ-C Buchse.
Beide Schnittstellen müssen so platziert sein, dass sie im eingebauten Zustand für den Benutzer zugänglich sind.
Rechts oben ist der ESP32 so platziert, dass die Antenne am Rand des PCBs liegt. Links und rechts von der Antenne bleibt ein kupferfreier Bereich von 5\,mm.

\begin{figure}[h]
	\centering
	\includegraphics[width=0.8\linewidth]{Layout.pdf}
	\caption{Übersicht des PCB-Layouts mit Unterteilung in Funktionsgruppen}
	\label{pic:Layout_Block}
\end{figure}
