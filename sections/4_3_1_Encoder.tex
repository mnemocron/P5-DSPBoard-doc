\subsubsection{Encoder Mode mit Hardware Timer}
\label{sec:Conf_Encoder}



\paragraph{Spezifikationen}

\begin{table}[H]
\begin{tabular}{|l|l|l|}
\hline
\textbf{Setting} & \textbf{Werte} & \textbf{Erklärung}                            \\ \hline
Counter Mode     & Up | Down      & Zählrichtung in Abhängigkeit der Drehrichung  \\ \hline
Counter Period   &                & maximaler Zählerwert (z.b. uint16\_t)         \\ \hline
Encoder Mode     & T1 | T2        & \begin{tabular}[c]{@{}l@{}}Triggerfokus auf CH1 oder CH2 oder beides.
                                     \\ Wenn beide aktiviert sind, zählt der Timer doppelt.\end{tabular} \\ \hline
\end{tabular}
\end{table}


\todo{SB - Bild von CubeMX}

\paragraph{Anwendung im Code}

\begin{lstlisting}[language=c]
/* USER CODE BEGIN 2 */
HAL_TIM_Encoder_Start(&htim2, TIM_CHANNEL_1);   // start Encoder mode on one channel
/* USER CODE END 2 */
\end{lstlisting}

\begin{lstlisting}[language=c]
/* USER CODE BEGIN x */
int new_encoder_val = TIM2->CNT;   // read encoder count anywhere in the code 
/* USER CODE END x */
\end{lstlisting}




