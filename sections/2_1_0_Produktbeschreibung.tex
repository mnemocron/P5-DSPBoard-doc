%\subsection{Projektziele}
%\label{sec:Projektziele}
%
%In Zusammenarbeit mit dem betreuenden Fachdozenten Prof. Dr. Markus Hufschmid, welcher auch gleichzeitig Auftraggeber ist, wurden folgende Projektziele und Anforderungen an das DSP-Board erarbeitet.
%
%
%\begin{table}[H]
%	\centering
%	\begin{tabular}{|l|l|}
%		\hline
%		\textbf{Nr.} & \textbf{Ziel}                                                                                     \\ \hline
%		1.1         & \begin{tabular}[c]{@{}l@{}}Microcontroller mit\\   Cortex-M4(F) Architektur mit FPU\end{tabular}   \\ \hline
%		1.2         & Audio Passthrough von Line-In nach Line-Out                                                        \\ \hline
%		1.3         & \begin{tabular}[c]{@{}l@{}}Audio Schnittstelle (analog)\\   - Line-IN\\   - Line OUT\end{tabular}  \\ \hline
%		1.4         & 2 Stk. Drehencoder für HMI                                                                  \\ \hline
%		1.5         & 2 Stk. Taster für HMI                                                                             \\ \hline
%		1.6         & \begin{tabular}[c]{@{}l@{}}1 Display zur Anzeige des\\   Funktionsmodus\end{tabular}               \\ \hline
%		1.7         & \begin{tabular}[c]{@{}l@{}}Microcontroller ohne Debugger\\   über USB programmierbar\end{tabular} \\ \hline
%	\end{tabular}
%	\caption{Harte Ziele aus dem Pflichtenheft}
%	\label{tab:harteZiele2}
%\end{table}
%
%\begin{table}[H]
%	\centering
%	\begin{tabular}{|l|l|l|}
%		\hline
%		\textbf{Nr.} & \textbf{Ziel}                                                                                                                                                   \\ \hline
%		2.1         & \begin{tabular}[c]{@{}l@{}}Anzahl Layer der Leiterplatte\\$n_{Layer} \leq 2$                                                                                                                                    \end{tabular} \\ \hline
%		2.2         & \begin{tabular}[c]{@{}l@{}}Bauteile sind handbestückbar\\   - SMD passiv $\geq$ 0603\\   - SMD cases: keine QFN / BGA\end{tabular}                                 \\ \hline
%		2.3         & \begin{tabular}[c]{@{}l@{}}Stromverbrauch erlaubt Betrieb über\\   USB 2.0 Speisung\\$I_{USB} =\leq 0.5\si{A}$\end{tabular}                                                                \\ \hline
%		2.4         & \begin{tabular}[c]{@{}l@{}}Audio Schnittstelle (analog)\\   - Headphone OUT\\   - Microphone IN\end{tabular}                                                   \\ \hline
%		2.5         & \begin{tabular}[c]{@{}l@{}}Audio Verbindung (digital)\\   - IN/OUT Board-to-Board Kommunikation\\   (mehrere Boards können\\   kaskadiert werden)\end{tabular}     \\ \hline
%		2.6         & Akkubetrieb möglich                                                                                                                                             \\ \hline
%		2.7         & \begin{tabular}[c]{@{}l@{}}zusätzliche (farbige) LEDs als\\   Anzeige des Betriebsmodus\end{tabular}                                                             \\ \hline
%	\end{tabular}
%	\caption{Weiche Ziele aus dem Pflichtenheft}
%	\label{tab:weicheZiele2}
%\end{table}
%Die harten Ziele \ref{tab:harteZiele2} sind Mindestanforderungen welche unbedingt einzuhalten sind, während die weichen Ziele \ref{tab:weicheZiele2} optional sind. Nachfolgend ist das erarbeitete Lösungskonzept näher erklärt welches zur Erreichung der Projektziele führen sollte.