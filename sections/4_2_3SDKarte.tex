\subsubsection{SD-Karte}
\label{sec:SDKarte}

%Die SD Karte stellt ein wichtiges Element des HMIs dar.
%Über die SD-Karte können die Gcode-Daten am schnellsten auf den Drucker aufgespielt werden.
%Die Position der SD-Karte ist leicht zugänglich an der Vorderseite des Druckers.
%Bevor eine Datei auf die SD-Karte aufgespielt werden kann, muss die SD-Karte im Menu von Marlin ausgeworfen werden.
%\todo[inline]{SB - Sollte es hier nicht heissen "Bevor die SD-Karte entfernt wird, um weitere Dateien daraufzuschreiben, muss sie im Menü von Marlin ausgeworfen werden. - A: MR}


Wie bereits erwähnt, dient die SD-Karte als Speicher für die G-Code Dateien. Die Firmware Marlin liest während des Druckvorgangs sukzessiv einige G-Code Befehle von der Datei auf der SD-Karte und rechnet diese in Motorbewegungen um. Des Weiteren wird das EEPROM auf der SD-Karte emuliert, da der Mikrocontroller über kein internes EEPROM verfügt. Dies bedeutet, dass sämtliche vom Benutzer eingestellte Parameter des Druckers auf der SD-Karte in der Datei \texttt{eeprom.dat} abgespeichert werden. Dies hat den Nachteil, dass bei Verwendung einer andern SD-Karte sämtliche Parameter nicht mehr vorhanden sind. In diesem Fall wird auf die Standardwerte zurückgegriffen. Ein Vorteil besteht darin, dass es so einfacher ist, zwischen verschiedenen Konfigurationen zu wechseln. Dazu muss lediglich die entsprechende Datei auf der SD-Karte ersetzt werden.

Speziell an der Implementierung der SD-Karte ist, dass die SDIO-Schnittstelle verwendet wird. Herkömmliche 3D-Drucker benutzen in den meisten Fällen die SPI-Schnittstelle. SDIO ist wesentlich schneller und erweitert die SPI-Schnittstelle um zusätzliche Funktionen wie z.B. Plug and Play \cite{SDIO}. Des Weiteren gibt dies der Druckersteuerung erweiterte Zukunftssicherheit.

Es besteht allerdings ein Flaschenhals beim Dateiupload über den ESP32. Die Geschwindigkeit beträgt lediglich $\approx\,2\,$kB/s. Der Grund dafür liegt im Protokoll, welches von der Firmware Marlin verwendet wird. Dieses ist nicht für den Dateiupload optimiert. Jede Zeile wird mit Zeilennummer und Checksumme versehen und dann Zeichen für Zeichen über die serielle Schnittstelle übertragen. Danach muss auf eine Bestätigung von der Firmware Marlin gewartet werden, bis die nächste Zeile übertragen werden darf. Somit ist der Dateiupload über das Webinterface für grössere Dateien (>\,60\,kB) ungeeignet. Eine Anpassung des Protokolls ist durchaus machbar. Dies liegt allerdings nicht im Rahmen dieses Projektes.
