\subsection{Firmware Marlin}
\label{sec:FirmwareMarlin}

In diesem Unterkapitel wird die 3D-Drucker Firmware Marlin beschrieben. Es wird auf getätigte Änderungen und vorgenommene Konfigurationen eingegangen.

Marlin ist ein unter GPLv3 lizensiertes Open-Source-Projekt mit dem Ziel, eine anpassbare und zuverlässige Firmware für 3D-Drucker zu sein \cite{GnuGPLv3}. Die offizielle Referenzplattform der Firmware ist die AVR-Mikrocontroller-Familie.
Zum Zeitpunkt dieses Projektes bestehen Bestrebungen, die Firmware für 32-Bit-Architekturen zugänglicher zu machen. Damit nicht von Vorne begonnen werden muss, wird dazu eine sogenannte Hardwareabstraktionsschicht (HAL) eingesetzt. Diese sorgt dafür, dass anstelle von AVR-Systemaufrufen, prozessorspezifische Systemaufrufe vorgenommen werden. Somit bleibt der Ablauf der Firmware gleich, es ändert sich lediglich, wie auf prozessorspezifische Ressourcen zugegriffen wird. Diese Weiterentwicklungen finden alle im Marlin \texttt{bugfix-2.0.x} Entwicklungsbranch statt. Die folgende Dokumentation bezieht sich auf die Struktur und die Funktionen dieses Branches.

Die erwähnten Anpassungsmöglichkeiten von Marlin finden in erster Linie in den beiden Dateien \texttt{Configuration.h} und \texttt{Configuration\_adv.h} statt. Im ersteren werden die groben Funktionen der Firmware wie die Schrittmotortreiber, SD-Karte und der spezifische LCD-Typ konfiguriert. In der zweiten Datei sind erweiterte Konfigurationen möglich.
Dort können die PID-Regelung und die Schrittmotortreiber parametrisiert werden. Bei den Motortreibern kann die Mikroschrittanzahl sowie der Strom eingestellt werden.

Um ein neues Controllerboard in Marlin zu integrieren, bedarf es einer Benennung des Boards sowie der Erstellung einer \texttt{pins\_BOARDNAME.h} Headerdatei. In dieser wird die Pinbelegung vom Mikrocontroller festgelegt \cite{MarlinBoards}.
