\subsection{Use Case}
\label{sec:UseCase}

Sobald ein 3D-Modell im CAD-Programm fertig gezeichnet ist, wird die Datei dem Slicer-Programm übergeben. Dieses sorgt dafür, dass das 3D-Modell in einzelne Druckschichten aufgeteilt wird und erzeugt den für den Drucker lesbaren G-Code. Um dieses G-Code File nun drucken zu können, wird es auf die SD-Karte des Druckers geladen. 

Bei der ersten Inbetriebnahme des Druckers muss das Druckbett ausgerichtet und das Filament eingefädelt werden. Wie dies genau funktioniert ist dem Benutzerhandbuch (Kapitel 1, Getting Started) zu entnehmen. Ist der Drucker eingerichtet, kann das G-Code File mittels das Display des Druckers, von der SD-Karte oder direkt über die Weboberfläche geladen und der Druck gestartet werden (Benutzerhandbuch Kapitel 1.2, G-Code laden). Vor jedem Druckvorgang werden die Achsen des Druckers neu referenziert. Dazu fahren die Achsen gegen den Endanschlag. Sobald der Motor blockiert, erkennt dies der Motorentreiber aufgrund seiner StallGuard Funktion (siehe Kapitel \ref{sec:StallGuardRueckinduzierteSpannung}). Anschliessend startet der Druck und der Druckfortschritt wird laufend auf dem Display und der Weboberfläche aktualisiert. Sobald dieser Fortschrittsbalken ausgefüllt ist, ist der Druckvorgang beendet.

Die magnetische Heizplatte wird vom Drucker entfernt und das Modell kann davon gelöst werden. Falls das gedruckte Modell Stützkonstruktionen verwenden muss, werden diese entfernt. Das fertig gedruckte Bauteil ist bereit für den Gebrauch.

Nach jedem Druckvorgang muss der Drucker wieder für die nächste Benutzung vorbereitet werden. Dies beinhaltet beispielsweise das Reinigen der Heizplatte oder das Nachfüllen des Filaments. Der genaue Vorgang ist dem Benutzerhandbuch zu entnehmen (Kapitel 1.4, Nach dem Druck).