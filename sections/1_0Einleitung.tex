\section{Einleitung}
\label{sec:Einleitung}

3D-Drucker sind heutzutage in der Industrie und im Hobbybereich weit verbreitet. Gerade in der Prototypenentwicklung oder bei der Herstellung von Massanfertigungen ermöglicht diese Technologie eine hohe Flexibilität. Auch in einem Studiengang, wo ständig neue Hardware entwickelt wird, kann ein 3D-Drucker hilfreich sein, um Gehäuse und mechanische Teile zu drucken.

Verschiedenste Hersteller bieten fortlaufend bessere und vor allem günstigere Fused Deposition Modeling (FDM) Drucker an. Doch die Qualität der Modelle aus Asien leiden unter dem Preisdruck. Diese Drucker haben oft nur billige Motortreiber und unterdimensionierte MOSFETs. Unter diesen Bedingungen wird einerseits die Sicherheit vor Brandgefahr \cite{JohnCogge} vernachlässigt und andererseits die Druckqualität dem Preis untergeordnet. Da die mechanischen Teile eines Druckers aus der mittleren Preisklasse jedoch robust sind, können diese beibehalten werden. Den grössten Gewinn in Punkto Sicherheit, Langlebigkeit und Funktionsumfang erhält man durch Auswechseln der Elektronik in Form der Steuerplatine.

Aus diesem Grund war es die Aufgabe im Projekt 4 des Studiengangs Elektro- und Informationstechnik einen existierenden 3D-Drucker zu optimieren.

Ziel dieser Arbeit ist die Planung, Bestückung, Programmierung und Inbetriebnahme einer Steuerplatine für einen 3D-Drucker-Bausatz, sodass die oben genannten Mängel aufgehoben werden. Die Steuerplatine soll die folgenden Funktionen umfassen: das Interpretieren des G-Codes von einer SD-Karte, das Ansteuern der Schrittmotoren der drei Achsen und des Filamentnachschubs sowie die Erkennung des Nullpunktes jeder Achse, sodass Aufträge gedruckt werden können. Des Weiteren soll die Platine die Temperatur des Extruders und des Heizbettes messen und regeln. Zur Steuerung kann auf bestehende Firmware-Lösungen zurückgegriffen werden, die bei Bedarf angepasst werden. Die Platine soll über ein Display bedient und alternativ über eine Wireless-Verbindung von einem Computer gesteuert werden.\

Zur Umsetzung der genannten Ziele wird eine Steuerplatine entwickelt, die von einem integrierten 32-Bit Mikrocontroller \textit{STM32} mit der Firmware \textit{Marlin} gesteuert wird. Für die Wirless Kommunikation mit dem PC wird ein Mikrocontroller \textit{ESP32} implementiert.
Neben dem Kommunikationsteil ist ein Leistungsteil auf der Platine enthalten, in dem die Motoren der Achsen und des Filamentvorschubs mittels Motorentreiber angesteuert und die Heizung des Extruders und des Heizbettes betrieben werden. Vom bestehenden Drucker Ender 3 Pro werden folgende Komponenten übernommen: Mechanik, Motoren, Lüfter, Heizbett, Extruder, Netzteil, Display, Temperatursensoren.

Die Entwicklung dieser Steuerplatine beinhaltet einen Hard- und einen Softwareteil. Aus diesem Grund ist der Hauptteil dieser Arbeit auch in diese beiden Kategorien aufgeteilt. Dem voran geht ein Kapitel (Kapitel \ref{sec:Gesamtuebersicht}), das eine Gesamtübersicht anhand des detaillierteren Lösungskonzepts bietet, sowie ein Kapitel, das die technischen Grundlagen erläutert und dem Leser so einen Einstieg in die Thematik des 3D-Drucks bietet (Kapitel \ref{sec:TechnischeGrundlagen}). Der darauffolgende Hardwareteil umfasst eine Beschreibung des bestehenden Druckers, des Gehäuses, der Filament Erkennung, sowie des Schemas und des Layouts der Steuerplatine (Kapitel \ref{sec:Hardware}). Die Softwaredokumentation in Kapitel \ref{sec:Software/Firmware}  umfasst eine Beschreibung der verwendeten Firmware \textit{Marlin}, des Human Machine Interfaces (HMI) sowie der Updatemöglichkeiten.
Den Abschluss des Hauptteils bildet das Kapitel \ref{sec:Validierung} Validierung. Darin werden die Überprüfung der Teilsysteme, ein Funktionstest und die Zielerreichung behandelt. Als Ausblick ist darin auch ein Kapitel enthalten, das über die Erweiterungsmöglichkeiten Aufschluss bietet.



