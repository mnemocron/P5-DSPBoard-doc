\vspace{3mm}
\label{sec:FilamentErkennung}
\paragraph{Filament Erkennung}
Um festzustellen, ob noch Filament vor dem Extruder vorhanden ist, wird eine Filament-Erkennung eingesetzt. Diese besteht aus einem selbst entworfenen und 3D-gedruckten Gehäuse und einem Endschalter mit einer Rolle am Auslöser. Grafik \ref{pic:FilamentErkennung} zeigt das Funktionsprinzip des Sensors.\\

Die Filament-Erkennung funktioniert grundsätzlich so, dass das Filament einen Endschalter betätigt und somit schliesst. Sobald kein Filament mehr vorhanden ist, wird dieser Schalter geöffnet. Da der Endschalter eine Rolle am Auslöser hat, wird so die Reibung mit dem Filament reduziert. Beim Entwerfen vom Gehäuse wurde darauf geachtet, dass Filament mit einem Durchmesser von 1.75 mm verwendet wird. Ausserdem sollte die Erkennung vor dem Extruder stattfinden, da sonst der Extruder versuchen würde das Filament vorzuschieben, auch wenn keines mehr vorhanden ist. Befestigt wird der Sensor direkt auf der X-Achse.

\begin{figure}[h]
	\centering
	\includegraphics[width=0.7\linewidth]{limit.eps}
	\caption{Diese Grafik zeigt den Aufbau der Filament-Erkennung. Im linken Teil ist kein Filament (blau) vorhanden und der Schalter (schwarz) ist geöffnet. Im rechten Teil ist Filament vorhanden, welches den Schalter herunterdrückt und ihn somit schliesst.}
	\label{pic:FilamentErkennung}
\end{figure}
