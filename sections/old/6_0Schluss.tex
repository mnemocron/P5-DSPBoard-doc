\section{Schluss}
\label{sec:Schluss}
\todo[inline]{Gegenlesen - Fertig - MR -  In diesem Kapitel steht die Zusammenfassung des Resultats und inwiefern die Projektziele damit erfüllt sind (mit Verweis auf die Validierung). Ebenfalls wird das Weiterentwicklungspotential aufgezeigt. - A: MR}

Anhand der Mechanik des 3D Druckers Ender 3 Pro und der eigens gefertigten Steuerung konnte ein robuster, leistungsfähiger und einfach zu bedienender Drucker entwickelt und in Betrieb genommen werden (siehe Kapitel \ref{sec:Validierung}). Neben allen definierten Sollzielen konnten auch alle Wunschziele umgesetzt werden.

Der Benutzer kann seine G-Code Files nun nicht nur per SD-Karte, sondern auch per Fernzugriff über das Webinterface auf den Drucker laden und den Druck auch gleich starten. Über den Druckfortschritt wird er auf dem Display des Druckers oder auch auf dem Webinterface informiert. Die beiden horizontalen Achsen des Druckers (X- und Y-Achse) werden ohne Endschalter referenziert. Dadurch ist die Anzahl der  Sensoren reduziert.
Der verwendete 32-Bit Prozessor verfügt über genug Rechenleistung, sodass zukünftig auch Erweiterungen vorgenommen werden können, die rechenaufwändig sind. Des Weiteren können Updates der Firmware über das Webinterface getätigt werden. Ungenutze Pins des Prozessors sind auf Headers geführt, wodurch die Implementierung eines weiteren Motors einfach umzusetzen ist und kaum Hardwareanpassungen benötigt werden.

Die verwendete Firmeware, Marlin, kommt im Hobbybereich häufig zum Einsatz, was der Realisierung dieses Projekts zu Hilfe kam. Für diverse Herausforderungen konnten online Lösungen gefunden  oder gar Unterstützung durch die Entwickler in Anspruch genommen werden. Für den professionellen Gebrauch sei an dieser Stelle aber von Marlin abzuraten, weil es sich um eine Open-Source-Software handelt, deren Übersichtlichkeit zu Wünschen übrig lässt.

Alles in Allem darf gesagt werden, dass das Projekt 4 erfolgreich durchgeführt werden konnte und der Drucker nach Abschluss des Projekts mit Stolz der FHNW übergeben wird.