\subsection{Lösungskonzept}
\label{sec:Lösungskonzept}

Das Lösungskonzept beschreibt, wie die Steuerung des Druckers Ender 3 Pro optimiert wird.

In Abbildung \ref{pic:BlockschaltbildLoesungskonzept} ist das Blockdiagramm des Lösungskonzepts ersichtlich. Die dünne, gestrichelte Linie stellt den 3D-Drucker als Ganzes dar. Der Benutzer kann diesen von aussen bedienen. Die dickere, gestrichelte Linie stellt die Systemgrenze dar. Alles was sich innerhalb dieser Grenzen befindet, wird in diesem Projekt realisiert. Ausserhalb der Systemgrenze befindet sich die Druckermechanik, welche vom bestehenden Drucker Ender 3 Pro übernommen ist.
Der Benutzer kann G-Code Files per SD-Karte oder Webinterface auf den Drucker laden (in Gelb - Kommunikation). Diese Files können entweder vom Webinterface oder vom im Drucker integrierten Display gestartet werden (in Grün - HMI). Als Mikrocontroller kommt ein 32-Bit Modell zum Einsatz, weil er mehr Rechenleistung für zukünftige Erweiterungen zur Verfügung stellt. Auf dem Mikrocontroller läuft die 3D-Drucker Firmware Marlin. Diese interpretiert den G-Code und steuert die Druckermechanik anhand der Treiber im Leistungsbereich (in Rot - Leistung). Zu den Treibern zählen die Motoren-, Heizungs- und Lüftertreiber. Um die Zuverlässigkeit des Druckers zu steigern, kommen hochwertige Motorentreiber zum Einsatz (siehe Kapitel \ref{sec:SteppertreiberTMC2660}). Im Sensorik-Block (in Blau) sind die zum Ender 3 Pro gehörigen Endschalter für die beiden horizontalen Achsen des Druckers (X- und Y-Achse) durch eine Blockiererkennung des Motorentreibers ersetzt (siehe Kapitel \ref{sec:StallGuardRueckinduzierteSpannung}). Eine Nachlauferkennung stellt sicher, dass nur gedruckt wird, wenn Filament vorhanden ist (siehe Kapitel \ref{sec:FilamentErkennung}. Die Temperatursensoren überwachen die Temperatur des Heizbettes und des Extruders.

Für genauere Informationen zum Lösungskonzept sei hier auf das Pflichtenheft im Anhang \ref{app:Pflichtenheft} verwiesen.
%
%Um den üblichen Gebrauch des 3D-Druckers besser verstehen zu können, ist im folgenden Kapitel ein typischer Use Case beschrieben.
 

\begin{figure}
	\centering
	\includegraphics[width=0.85\linewidth]{Systemgrenzen.eps}
	\caption{Blockschaltbild des Lösungskonzepts}
	\label{pic:BlockschaltbildLoesungskonzept}
\end{figure}


