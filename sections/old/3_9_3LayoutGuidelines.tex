\subsubsection{Layout-Richtlinien}
\label{sec:LayoutGuidelines}

Nachfolgend sind die Layout-Richtlinien der wichtigsten Bauteile erwähnt.\\


\paragraph{Schrittmotortreiber TMC2660}
Trinamic hat wichtige Hinweise und Vorgaben zum Layoutdesign des Schrittmotortreiber im Datenblatt des TMC2660 aufgeführt \cite{TMC2660}.
Die wichtigsten Punkte sind:

\begin{itemize}
	\item Die Verbindungen der beiden Strommesswiderstände dürfen keinen gemeinsamen Massepfad aufweisen und sollen möglichst kurz gehalten werden
	\item Die Ausgänge \texttt{OA} und \texttt{OB} sind für den Wärmetransfer zuständig und benötigen grosse Kupferflächen
	\item Eine Massefläche über das ganze Board sorgt für ausreichend Kühlung und Abschirmung
\end{itemize}

Des Weiteren ist im Datenblatt ein Referenz-Layout aufgeführt. Dieses konnte weitestgehend übernommen werden konnte.


\paragraph{Spannungsregler LM2675M}
Texas Instruments hat mehrere Empfehlungen für das Layout der Spannungsregler und stellt ein Referenz-Layout zur Verfügung \cite{LM2675M}. Die wichtigsten zwei Punkte sind nachfolgend aufgelistet:

\begin{itemize}
	\item Externe Komponenten nahe am Spannungsregler platzieren (kurze Verbindungen)
	\item Alle stromführenden Leitungen (inkl. Masseverbindungen) benötigen ausreichende Kupferflächen für niederohmige Verbindungen
\end{itemize}

\vspace{3mm}
\paragraph{Leiterbahnbreite Extruder und Heizbett}
Da die 24\,V Leiterbahnen für den Extruder und das Heizbett 2\,A resp. 8\,A führen, müssen diese genügend breit gelayoutet werden.
Die Kupferdicke beträgt 35\,$\si{\mu m}$ für die Aussenlagen resp. 12\,$\si{\mu m}$ für die Innenlagen.
Die längste Verbindung hat eine Länge von 42\,mm und führt 2\,A zum Extruder.
Es wurde mit einer maximalen Temperaturerhöhung von 25\,\textdegree C und einem Strom von 8\,A gerechnet.
In Tabelle \ref{tab:Leiterbahnen_Calc} sind sämtliche Parameter ersichtlich.
Die vorgeschlagene, minimale Leiterbahnbreite beträgt 3.03\,mm.
Im Layout wird die Verbindung auf allen vier Lagen der Leiterplatte mit einer Breite von 2.5\,mm geführt.
Für die Top- und Bottom-Lage addiert sich die Leiterbahnbreite zu 5\,mm.
Ausserdem sind alle weiteren Leiterbahnen kürzer und/oder führen nur 2\,A im Gegensatz zu den 8\,A.
Die Temperaturvorgabe ist somit in jedem Fall gewährleistet.
Die Berechnungen stammen aus dem Tool von \textit{4pcb.com} \cite{4PCB}.\\

\begin{table}[h]
\small
	\begin{center}
	\def\arraystretch{1.3} 
		\begin{tabular}{|C{3.5cm}|c||C{3.5cm}|c|c|}
			\hline
			 \textbf{Eingabeparameter}  & \textbf{Wert} & \textbf{Ausgabeparameter} & \textbf{Wert Innenlage} & \textbf{Wert Aussenlage} \\ \hline 
			Strom & $ 8 \,\si{A}$  & Leiterbahnbreite & $7.89 \,\si{mm}$	& $3.03\,\si{mm}$	\\ \hline
			 Kupferdicke & $35\,\si{\mu{}m}$& Verlustleistung & $182\,\si{mW}$ & $475\,\si{mW}$\\ \hline
			 Temperaturerhöhung & $25\,\si{\degree C}$ &  Spannungsabfall & $22\,\si{mV}$ & $59\,\si{mV}$ \\ \hline
			 Umgebungs- temperatur & $25\,\si{\degree C}$ & Widerstand & $2.8\,\si{m\Omega}$ & $7.4\,\si{m\Omega}$ \\ \hline
			 Verbindungslänge &   $42 \, \si{mm}$ & & & \\ \hline
		\end{tabular} 
	\end{center}
	\caption{Berechnete Parameter für die Leiterbahnen}
	\label{tab:Leiterbahnen_Calc}
\end{table}

\newpage
\paragraph{USB-Datenleitungen}
Die USB-Datenleitungen haben eine typische Impedanz von $90\,\si{\Omega}$ und müssen aufgrund der Signallaufzeit und der Störsicherheit differentiell geführt werden. Darum sind die Netznamen der Datenleitungen im Schema mit \texttt{\_N} resp. \texttt{\_P} bezeichnet, um die Routing-Hilfe von KiCad in Anspruch zu nehmen.
