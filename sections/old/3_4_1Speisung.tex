\subsubsection{Speisung}
\label{sec:Speisung}


Auf dem Print wird eine 3.3$\,\si{\volt}$ und eine 5$\,\si{\volt}$ Speisung benötigt. In diesem Kapitel wird erläutert, wie diese dimensioniert, gesichert und mit einem Verpolschutz versehen wurden.

\paragraph{Spannungsregler für 3.3V und 5V}
Nachfolgend sind tabellarisch die Anforderung an die beiden DC-DC-Wandler aufgelistet:

\begin{table}[H]
\title{Stromverbrauch 3.3V}
\centering
\def\arraystretch{1.1} 
\begin{tabular}{|l|c|l|}
\hline
\textbf{Beschreibung} & \textbf{max. Strom in mA} & \textbf{Quelle} \\ \hline
STM32          & 150                  & \cite[p. 43 - Table 8]{STM32F103RE} \\ \hline
Oszillator 8\,MHz & 1.8                  & \cite{EpsonCrystal} \\ \hline
SD Karte       & 200                  & \cite{KingstonSD} \\ \hline
Pegelwandler & 2 x 100               & \cite[p. 6 - Table 5]{NXPLevelShift} \\ \hline
ESP32-WROOM-32 & 80                   & \cite{ESP32Wroom} \\ \hline
LEDs           & 10 x 5                 & \\ \hline
\textbf{Total}          & \textbf{681.8}               & \\ \hline
\end{tabular}
\end{table}

\begin{table}[H]
\title{Stromverbrauch 5V}
\centering
\def\arraystretch{1.1} 
\begin{tabular}{|l|c|l|}
\hline
\textbf{Beschreibung} & \textbf{max. Strom in mA} & \textbf{Quelle} \\ \hline
FT2232D        & 95                   & \cite[p. 16 - Table 5.1]{FT2232D} \\ \hline
Oszillator 	$6\,\si{MHz}$   & 3         & \cite{EpsonCrystal6MHz} \\ \hline
Display        & 40                   & \cite{12864LCD} \\ \hline
\textbf{Total}          & \textbf{138}                  & \\ \hline
\end{tabular}
\end{table}

Verbaut sind zwei LM2675M Step-Down Converter (U101 / U102) mit einem maximalen Ausgangsstrom von 1\,A. Dadurch ist bei 3.3\,V auch gewährleistet, dass eine Reserve für den ESP32 (beispielsweise während des Verbindungsaufbaus) vorhanden ist. Das Datenblatt des ESP32 verlangt, dass die Versorgung darauf dimensioniert sein soll, 500\,mA zu liefern \cite[S. 10]{ESP32}.\\


\paragraph{Sicherungen}
Da ein Kurzschluss an einer mit $24\,\si{V}$ gespeisten Komponente, wie beispielsweise dem Heizbett, den Print zerstören könnte, sind diese separat abgesichert. Die Sicherungswerte richten sich dabei nach deren Stromverbrauch und sind nachfolgend aufgeführt. Ebenfalls wurde die Einspeisung mit einer 20\,A Sicherung geschützt. Die Sicherungen für das Hotend und das Heizbett sind im \texttt{appendix\_SENSORS\_POWER.sch} zu finden. Verbaut sind Sicherungen vom Typ ATO, auch bekannt als Autosicherungen. Diese sind einfach verfügbar, robust, benutzerfreundlich und auch für grosse Ströme verfügbar. 

\begin{table}[H]
\centering
\def\arraystretch{1.1} 
\begin{tabular}{|l|c|c|c|}
\hline
\textbf{Beschreibung} & \textbf{Widerstand in $\Omega$} & \textbf{Strom in A} & \textbf{Absicherung in A}  \\ \hline
Heizbett     & 3                    & 8             & 10              \\ \hline     
Hotend         & 12                   & 2             & 3               \\ \hline
24\,V Input      & -                    & max. 14.6              & 20              \\ \hline 
\end{tabular}
\end{table}

\vspace{3mm}
\paragraph{Verpolschutz}
Da die 24\,V Schraubklemme keinen mechanischen Verpolschutz gewährleistet, ist die Einspeisung mit einem Verpolschutz versehen. Dazu wurde ein n-Kanal-MOSFET (T101) in die Masseleitung geschaltet. Die Schaltung funktioniert bei korrekter Polung so, dass im Einschaltmoment die parasitäre Body-Diode des MOSFETs den Stromkreis schliesst. Dadurch liegt am Gate des MOSFETs die Zener-Spannung der Z-Diode (D105) von 7.5\,V an. Das führt wiederum dazu, dass der n-Kanal-MOSFET in den leitenden Zustand übergeht. Wird die Einspeisung verpolt angeschlossen, so wird die parasitäre Body-Diode des MOSFETSs gar nicht erst leitend und der Stromkreis ist somit unterbrochen. Die gesamte Spannug liegt nun am MOSFET zwischen Drain und Source an, da sich dieser wie ein geöffneter Schalter verhält.
Die Anforderungen an den MOSFET sind demzufolge, dass dieser eine Drain-Source-Spannung $U_{DS}$ von 24\,V schalten kann und einen sehr kleinen Drain-Source-Widerstand $R_\text{DSon}$ aufweist. Es wäre grundsätzlich auch möglich einen p-Kanal-MOSFET in die $24 \,\si{V}$ Leitung zu schalten. Aus den unten genannten Gründen wurde jedoch darauf verzichtet.

Die Vorteile der n-Kanal-MOSFET Schaltung sind: 
\begin{itemize}
	\item N-Kanal-MOSFETs haben zum gleichen Preis ein kleineres $R_\text{DSon}$ als p-Kanal-MOSFETs
	\item Um das Heizbett und das Hotend zu schalten werden bereits passende MOSFETs verwendet
\end{itemize}

Die Nachteile gegenüber einem p-Kanal-MOSFET Verpolschutz sind: 
\begin{itemize}
	\item Das Massepotenzial der Schaltung ist nicht mehr auf demselben Potenzial wie die Masse der 24\,V Einspeisung
\end{itemize}

In Abbildung \ref{pic:Schema_Verpolschutz} ist der eingesetzte Verpolschutz mit dem n-Kanal-MOSFET dargestellt.


Der genannte Nachteil spielt bei dieser Anwendung jedoch keine Rolle, da keine Kommunikation mit äusseren Geräten stattfindet, bei der ein gemeinsames Massepotenzial nötig ist. Jedoch muss bei Messungen an der Schaltung darauf geachtet werden, dass auf das richtige GND referenziert wird und keine Massekurzschlüsse erzeugt werden.\\
Weiter ist darauf zu achten, dass der gesamte Strom durch den MOSFET fliesst. Der gewählte n-Kanal-MOSFET, BUK963R3-60E, besitzt bei $7.5 \,\si{V}$ Gate-Source-Spannung ein $R_\text{DSon} <3.3 \,\si{m\Omega}$. Dies entspricht einer maximalen Verlustleistung von $0.7\,\si{W}$ @ $14.6\,\si{A}$ oder $0.2\,\%$ der Gesamtleistung des Druckers \cite{BUK963R3}. Der MOSFET erwärmt sich dann um ungefähr $35\,\si{\degree C}$ was ein Kühlkörper unnötig macht da $60\,\si{\degree C}$ Betriebstemperatur immer noch im regulären Bereich sind. Weiter werden diese wohl nie erreicht, da sich unter dem FET eine grosse Kupferfläche befindet die den thermischen Widerstand von $50\,\si{K/W}$ weiter reduzieren. 

Zusätzlich ist direkt am Eingang eine bidirektionale LED vorgesehen, die bei korrekter Polung grün und bei inkorrekter Polung rot leuchtet.
Zu beachten ist, dass bei $24\,\si{V}$ Speisung die max. Verlustleitung des Vorwiderstandes, $0.5\,\si{W}$ für 1206 Bauteile, nicht überschritten werden darf. Darum wurde bei der Auswahl der LED auf einen geringen Vorwärtsstrom geachtet. Mit $10\,\si{k\Omega}$ und max. $2.4\,\si{mA}$ ist das gegeben.

\begin{figure}[h]
	\centering
	\includegraphics[width=0.5\linewidth]{Verpolschutz_Schema.pdf}
	\caption{Schemaausschnitt vom Verpolschutz: Bei korrekt gepolter Speisung wird der MOSFET (T101) leitend und verbindet somit das Massepotenzial auf die Masse der 24\,V Einspeisung}
	\label{pic:Schema_Verpolschutz}
\end{figure}

\vspace{3mm}
\paragraph{Trennung der Analogen und digitalen Speisung}
Die 3.3\,V werden grundsätzlich von allen digitalen Bausteinen auf der Leiterplatte benötigt.
Um eine akkurate AD-Wandlung vornehmen zu können, braucht der STM32 zusätzlich eine gefilterte 3.3\,V Analogspeisung.
Diese wird mit einem separaten Netz (+3.3VA) realisiert und ist, wie in \ref{pic:Schema_AVCC} dargestellt, mit den ursprünglichen 3.3\,V über einen SMD-Ferrit (FB101) verbunden.
Ausserdem werden die 3.3\,V für die Logik I/O der Schrittmotortreiber verwendet. Die Schrittmotortreiber sind auf der Leiterplatte an einer Position, in welcher viel Leistung geschaltet wird.
Aus diesem Grund ist ein separates Netz (+3.3VP) vorgesehen, das ebenfalls über einen SMD-Ferrit (FB102) auf die 3.3\,V geführt ist. 
So wird vermieden, dass Störungen vom Leistungsteil in den digitalen Schaltungsteil einfliessen.

\begin{figure}[h]
	\centering
	\includegraphics[width=0.4\linewidth]{Schema_Analog_Supply_Filter.pdf}
	\caption{Schemaausschnitt der analogen und der digitalen Speisungstrennung}
	\label{pic:Schema_AVCC}
\end{figure}
