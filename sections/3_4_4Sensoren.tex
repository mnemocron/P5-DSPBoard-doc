\subsubsection{Sensoreingänge und Leistungstreiber}
\label{sec:SensoreingaengUndLeistungstreiber}

In diesem Kapitel sind die Schaltungen des Teilschemas \texttt{appendix\_SENSORS\_POWER.sch} beschrieben.
Dies beinhaltet die Schnittstellen der Peripherie vom 3D-Drucker und die Leistungsschalter der Lüfter und der Heizelemente.\\

\paragraph{ADC-Eingänge}
Die Abbildung \ref{pic:Schema_ADCin} zeigt die Eingangsbeschaltung der analogen Messeingänge für die Temperaturmessung. Diese müssen gemäss einer Vorgabe der Firmware Marlin einen $4.7\,\si{k\Omega}$ Pull-up Widerstand vorgeschaltet haben. Ausserdem werden hochfrequente AC-Anteile durch  zusätzliche $10\,\si{\mu F}$ Kondensatoren geblockt.
Da die Steckverbinder direkt auf einen Pin des STM32 führen und vom Benutzer berührt werden können, werden diese Eingänge mit einer Schutzdiode versehen (z.B. D605).

\begin{figure}[H]
	\centering
	\includegraphics[width=0.55\linewidth]{Schema_ADC_Input.pdf}
	\caption{Analoger Messeingang für einen 100\,k$\Omega$ NTC-Thermistor mit Vorwiderstand R401, Filter sowie Schutzdiode}
	\label{pic:Schema_ADCin}
\end{figure}

\paragraph{Referenzschalter}
Im Abschnitt \ref{sec:StallGuardRueckinduzierteSpannung} wird die Theorie zur Verwendung des StallGuards als Endschalter beschrieben. Da die StallGuard-Funktion jedoch nur als Referenzschalter für die Referenzfahrt der X- und der Y-Achse verwendet wird, siehe Kapitel \ref{sec:Konfiguration_Motortreiber}, sind die Stecker für konventionelle Endschalter weiterhin auf dem Board vorhanden.
Somit ist es weiterhin möglich, reguläre Endschalter für alle Achsen zu verwenden. Dies kann bei einem anderen 3D-Drucker nötig sein, wenn die Achsen nicht bis zum Anschlag fahren können oder wenn die Mechanik dafür nicht stabil genug ist (wie bei der Z-Achse vom Ender 3 Pro).

Die Endschalter vom Ender 3 Pro sind im Normalzustand geschlossen (NC). Dies hat den Vorteil, dass auch ein möglicher Kabelbruch erkannt werden kann. Um das Öffnen des Kontakts beim Erreichen eines Endanschlags zu erkennen, wird ein Pull-up Widerstand benötigt, welcher den GPIO-Pin auf \texttt{HIGH} zieht.
Der Schaltertyp (NO / NC) kann in der Firmware Marlin konfiguriert werden. Wobei das Benutzen eines normalerweise geöffneten Endschalters (NO) nicht sinnvoll ist, da ein Kabelbruch nicht erkannt werden könnte.
Auch die Eingänge der Endschalter sind mit entsprechenden Schutzdioden versehen.\\

\paragraph{Leistungsschalter Lüfter}
Der Ender 3 Pro hat drei 24\,V / 100\,mA Lüfter, welche angesteuert werden müssen. 
Der MOSFET muss mit 3.3\,V Gate-Source-Spannung voll durchgeschaltet werden können, 24\,V Drain-Source-Spannung aushalten und 100\,mA Drainstrom führen können.
Deshalb kommt ein PMV40UN2 n-Kanal-MOSFET (T405) wie in Abbildung \ref{pic:Schema_Fan} zum Einsatz \cite{PMV40UN2}.
Dieser MOSFET bietet eine niedrige Einschaltschwelle und hat bei $U_{\text{GS}}=3\,\si{V}$ ein  $R_{\text{DSon}}$ von $0.05\,\si{\Omega}$.
Ausserdem ist die maximale Sperrspannung $U_{\text{DS, max}}=30\,\si{V}$ und die maximale Verlustleistung  $P_{\text{max}}=490\,\si{mW}$.
Diese wird mit  $P=0.05\,\si{\Omega}\cdot(0.1 \,\si{A})^2 = 500\,\si{\mu W}$ bei Weitem unterschritten.

Zusätzlich ist eine STPS1L30A Shottky Diode (D405) als Freilaufdiode verbaut, da der Lüfter eine induktive Last darstellt.

Da das Gate eines MOSFET eine Gatekapazität aufweist und der Mikrocontroller nicht unbegrenzt viel Strom liefern kann, muss ein minimaler Gatewiderstand berechnet werden.
Der maximale Ausgangsstrom von einem GPIO-Pin des STM32 beträgt $I_{\text{OUT, max}} = 25\,\si{mA}$.
Im Einschaltmoment ist die Gatekapazität ungeladen. Daraus folgt für den minimalen Gatewiderstand
\[
R_{\text{G, min}}=\frac{3.3\,\si{V}}{25\,\si{mA}}=132\,\si{\Omega}\ .
\]
%\vspace{3mm}
\begin{figure}[H]
	\centering
	\includegraphics[width=0.7\linewidth]{Schema_Fan.pdf}
	\caption{Schema eines Leistungstreibers für einen der drei Lüfter}
	\label{pic:Schema_Fan}
\end{figure}

\paragraph{Leistungsschalter Heizbett und Hotend}
Wie in Kapitel \ref{sec:Speisung} erwähnt, benötigen das Heizbett ca. 8\,A und das Hotend ca. 2\,A bei 24\,V. Um die Stückliste kleinzuhalten, wurde für beide Leistungsschalter derselbe MOSFET-Typ verwendet.
Dies bedeutet, der MOSFET muss im leitenden Zustand mindestens die o.\,g. 8\,A führen und im Sperrzustand 24\,V Drain-Source-Spannung aushalten. Ausserdem soll seine Schaltschwelle so liegen, dass er mit 3.3\,V Gate-Source-Spannung voll durchgeschaltet wird. Deshalb kommt ein BUK963R3-60E n-Kanal-MOSFET zum Einsatz. Dieser hat eine Spannungsfestigkeit von $U_{\text{DS}}=U_{\text{DG}}=60\,\si{V}$, einen Maximalstrom von $I_D=120\,\si{A}$ @ $U_{\text{GS}}=5\,\si{V}$ und eine maximale Verlustleistung von $P_{\text{max}}=293\,\si{W}$.
Der Drain-Source-Widerstand beträgt $R_{\text{DSon}}=3.75\,\si{m\Omega}$ @ $V_{\text{GS}}=3.3 \,\si{V}$ \cite{BUK963R3}.
Somit beträgt die Verlustleistung im leitenden Zustand $P=3.75\,\si{m\Omega}\cdot(8\,\si{A})^2 = 240\,\si{mW}$. Da derselbe FET wie für den Verpolschutz verwendet wird, ist auch hier kein Kühlkörper nötig, um die Verlustleistung abzuführen. \\

\paragraph{Servoanschlüsse}
Eine Funktion der Firmware Marlin erlaubt es, an sogenannten Servo-Pins weitere Hardware anzuschliessen. Dazu wurde jeweils ein GPIO-Pin an die Steckverbinder J605 und J606 geführt.
Wie der Name schon verrät, sind diese primär für den Anschluss von Servomotoren gedacht.
Es kann allerdings jede beliebige Hardware angeschlossen werden, welche nur einen GPIO-Pin benötigt. Eine Möglichkeit besteht darin, einen BLTouch Sensor anzuschliessen, mit welchem die Ebenheit des Druckbetts ausgemessen werden kann. So können beispielsweise Unebenheiten oder eine leichte Achsneigung kompensiert werden.
\vspace{5mm}