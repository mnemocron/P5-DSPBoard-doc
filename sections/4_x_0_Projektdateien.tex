\subsection{Projektstruktur}
\label{sec:SWProjekt}

Die Dateistruktur des Softwareprojektes in Keil uVision 5 ist unten beschrieben. 
Es sind nur die Dateien erwähnt, die für die Entwicklung der Software massgebend sind.

\todo{Verlinkungen auf Unterkapitel}

\paragraph{main.c}

In der \texttt{main.c} Datei befindet sich der Hauptteil des Softwareprojektes.
Libraries werden inkludiert, glbale sowie lokale Variablen deklariert und die Initialisierung des STM32 und aller Peripherie aufgerufen.
Innerhalb der \texttt{while(1)} Schlaufe ist eine State Machine realisierbar, die den Programmablauf des User Interfaces bestimmt.
Viele Variablen sind global und \texttt{volatile} deklariert und werden laufend von den Interrupt Handlern (IRQ) in \texttt{stm32f4xx\_it.c} verändert.


\paragraph{stm32f4xx\_it.c}

In dieser Datei sind die Interrupt Request Handler (IRQ) Funktionen ausprogrammiert.
Die definierten Funktionen werden bei verschiedenen Interrupt Events wie externen GPIO Interrupts (EXTI) oder DMA Transmission Completion Interrupt aufgerufen.


\paragraph{dsp\_board\_bsp.c}

Die Datei \texttt{dsp\_board\_bsp.} stellt ein Board Support Package für das P5 DSP Board dar.
In dieser Datei sind Funktionen ausgelagert, die spezifisch für die Hardware gemacht sind.
Die Interrupts werden hier abgefangen und über \texttt{extern volatile} Variablen an die State Machine im \texttt{main.c} weitergegeben.
Ausserdem sind gewisse, nicht spezifisch auf die Hardware ausgelegte, Helferfunktionen (z.B. Sinusgenerator) in dieser Datei ausgelagert.


\paragraph{ssd1306.c}

Die Dateien \texttt{ssd1306.c}, \texttt{ssd1306\_fonts.c} und \texttt{ssd1306\_tests.c} beinhalten die Funktionen zur Ansteuerung der SSD1306 OLED Displays und bilden die Library.


\paragraph{tlv320aic.c}

Die \texttt{tlv320aic.c} Datei bildet die Library für den TLV320 Audio Codec und beinhaltet Funktionen zur Lautstärkeregelung und Initialisierung des Codecs.


\paragraph{dsp\_processing.c}

In der DSP Processing Datei werden die empfangenen Audiodaten an verschiedene \\
DSP-Funktionen wie FIR-Filter verteilt und der Output-Buffer für den DMA Controller befüllt.


\paragraph{fir.c}

In dieser Datei ist ein FIR Filter aus der CMSIS/DSP Library implementiert.


\paragraph{adaptive\_fir.c}

Diese Hilfsbibliothek stellt eine Funktion ähnlich dem MATLAB Befehl \texttt{fir1()} zur Verfügung und wird benutzt um auf dem STM32 die FIR Tiefpassfilterkoeffizienten zu berechnen.
Die Datei beinhaltet ausserdem eine Window-Funktion, die von \texttt{fir1} benötigt wird.

