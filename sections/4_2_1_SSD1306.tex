\subsubsection{SSD1306 C Library}
\label{sec:Library_ssd1306}

Zur Ansteuerung der OLED Displays wird die Library stm32-ssd1306 von Aleksander Alekseev \ref{} verwendet.

\paragraph{Spezifikationen}

\begin{table}[H]
\title{Eigenschaften der STM32-SSD1306 Library}
\centering
\begin{tabular}{|l|c|}
\hline
\textbf{Beschreibung} & \textbf{Wert}     \\ \hline
Lizenz                & MIT               \\ \hline
RAM Bedarf            & 1 kiB pro Display \\ \hline
Textunterstützung     & ja                \\ \hline
Schriftarten          & 3 font sizes      \\ \hline
Grafikunterstützung   & nein              \\ \hline
\end{tabular}
\end{table}

Die Library funktioniert so, dass ein Pixelbuffer pro Display im RAM erstellt wird.
Der Buffer wird beim Aufruf der Funktion \texttt{ssd1306\_UpdateScreen()} über den I2C Bus auf das Display geschrieben.
Dadurch entsteht ein RAM Bedarf von:

${W * H / 8 = 128 * 64 / 8 = 1024\ \si{Bytes}}$

\paragraph{Änderungen an der Library}

Die Library unterstützt nur ein Display an einem I2C oder SPI Bus. Da bei diesem Projekt zwei Displays an unterschiedlichen Peripherischnittstellen sitzen, ist die Library für diese Anwendung angepasst.
Jeder Funktion muss nun ein Pointer auf einen Display Struct mitgegeben werden.
Im folgenden Listing ist dargestellt, wie die Library in C verwendet wird.

\begin{lstlisting}[language=c]
/* USER CODE BEGIN Includes */
#include "ssd1306.h"
/* USER CODE END Includes */
\end{lstlisting}

\begin{lstlisting}[language=c]
/* USER CODE BEGIN PV */
SSD1306_t holed1;    // Display Struct
/* USER CODE END PV */
\end{lstlisting}

\begin{lstlisting}[language=c]
/* USER CODE BEGIN 2 */
holed1.hi2cx = &hi2c1;   // set peripheral interface of Struct to I2C

ssd1306_Init(&holed1);
ssd1306_Fill(&holed1, Black);       // all pixels black
ssd1306_SetCursor(&holed1, 2, 0);   // x = 2px (from left) / y = 0px (from top)
ssd1306_WriteString(&holed1, "FHNW", Font_11x18, White); // medium font
ssd1306_UpdateScreen(&holed1);      // write Buffer to OLED Display
\end{lstlisting}

Die Library hat folgende Einschränkung: Alle Displays müssen entweder über I2C oder SPI Peripheriebusse angeschlossen sein.
Ein Mischen von SPI und I2C ist nicht möglich. Ausserdem sind die Funktionen für SPI nicht implementiert.

\paragraph{Copyright Notice}

\textbf{Copyright (c) 2018 Aleksander Alekseev}\
\
\
