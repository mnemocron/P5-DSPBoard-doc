\subsection{Konzept USB Akkuladeregler IC}
\label{sec:Konzept_Charger}

Ein weiches Ziel ist die Autonomie ohne externe Energieversorgung. 
Dazu soll ein Akkumulator genügend Energie liefern, um die Schaltung während einiger weniger Stunden (live-Konzert) zu betreiben. 
Der Ladestrom soll nicht grösser als die über USB-2.0 zugelassenen $2.0\si{A}$ betragen.

\begin{table}[H]
	\centering
	\begin{tabular}{|c|c|c|}
		\hline
		\textbf{Specification} & \textbf{BQ2409x}             & \textbf{MAX1811} \\ \hline
		Ladestrom              & 100mA / 500mA                & 100mA / 500mA    \\ \hline
		Ladespannung           & 4.2V                         & 4.1V / 4.2V      \\ \hline
		Features               & safety timer / thermal sense & thermal sense    \\ \hline
		Gehäuse                & MSOP-10                      & SO-8             \\ \hline
		Preis @ 25 Stk.        & 1.13                         & 3.90             \\ \hline
	\end{tabular}
	\caption{Parametervergleich zweier USB Akkuladungs-ICs}
	\label{tab:ComparisonCharger}
\end{table}

Wie in der Tabelle \ref{tab:ComparisonCharger} gezeigt, existieren integrierte Schaltungen für den sehr spezifischen Verwendungszweck der Akkuladeregelung ab USB Speisung.
Der BQ2409x überwiegt durch den niedrigen Preis und den grossen Funktionsumfang.
