\subsection{Codec}
\label{subsec:Konzept_Codec}

Der Codec bildet das analoge Frontend und muss mindestens die Verbindungen für Line-IN und einen Line-OUT vorweisen. Wünschenswert sind zudem ein MIC-IN und ein Headphone-OUT. Der Codec soll variable Samplingraten zwischen 8-96kHz unterstützen eine sinnvolle Auflösung (sicher über 10 bit) bieten. 

\begin{table}[H]
	\centering
	\begin{tabular}{|c|c|c|c|}
		\hline
		\textbf{Specification} & \textbf{TLV320AIC23B}             & \textbf{TSCS25A3} & \textbf{SGTL5000XNAA3R2} \\ \hline
		Auflösung              & 24 bit      & 32 bit          & 24 bit    \\ \hline
		Gehäuse           & TSOP  & QFN & QFN   \\ \hline
		SNR              & 90 / 100 dB & 90 / 102 dB &85 / 100 dB   \\ \hline
		Sampling                &8-96 kHz                      & 8-96 kHz&8-96 kHz             \\ \hline
		Preis         & 8.66                         & 7.45 &    2.96         \\ \hline
	\end{tabular}
	\caption{Parametervergleich der Audio-Codecs}
	\label{tab:codec}
\end{table}

Die drei verglichenen Audio-Codecs erfüllen alle die geforderten Qualitätsmerkmale, jedoch ist nur der \textbf{TLV320AIC23B}, wie vom Auftraggeber gefordert, handlötbar. Die anderen beiden anderen wären zwar günstiger, aber beide in einem QFN Gehäuse.