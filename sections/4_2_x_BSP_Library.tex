\subsubsection{Board Support Package (BSP) für das DSP Board}
\label{sec:LibBSP}

Für das DSP Board existiert in der Projektstruktur eine Library mit Funktionen speziell auf die Hardware des DSP Boards zugeschnittenen Funktionen.
Nachfolgend sind deren Funktion und Anwendung beschrieben.


\paragraph{Klinkenstecker Verbindungsstatus \texttt{BSP\_ReadJackConnected}}

Die Jack Verbindungsstecker haben einen Schalter eingebaut, dessen Zustand über einen GPIO am STM32 ausgelesen werden kann.
Nachfolgendes Listing zeigt ein Beispiel in C.

\begin{lstlisting}[style=Cuvision,caption={Verbindungsstatus des Line-In auslesen}]
if(BSP_ReadJackConnected(JACK_LINE_IN) == 1){
  // Line-In Jack is present / connected
} else {
  // No Line-In connected
}
\end{lstlisting}


\paragraph{Encoderwerte auslesen \texttt{BSP\_ReadEncoder\_Difference}}

Zum Auslesen der Encoder exisiteren zwei Funktionen.
\texttt{BSP\_ReadEncoder} gibt den absoluten Encoderwert zurück.
Interessanter für die Anwendung ist jedoch die Differenz seit dem letzten Mal auslesen.
Die Funktion \texttt{BSP\_ReadEncoder\_Difference} liefert einen signed (\texttt{int16\_t}) Wert zurück, um wie viele Ticks der Encoder in welche Richtung (Vorzeichen) gedreht wurde.
Positive Werte bedeuten im Uhrzeigersinn (CW), negative Werte stehen für Gegenuhrzeigersinn (CCW). \\
 \\
 
\begin{lstlisting}[style=Cuvision,caption={Auslesen wie weit und in welche Richtung der linke Encoder gedreht wurde.}]
int16_t encoder_change = BSP_ReadEncoder_Difference(ENCODER_LEFT);
\end{lstlisting}


\paragraph{Akkuspannung messen \texttt{BSP\_ReadBatteryVoltage}}

Die Funktion \texttt{BSP\_ReadBatteryVoltage} liefert einen \texttt{float} Wert für die über den ADC gemessene Akkuspannung zurück. 
Die Genauigkeit dieser Art der Messung ist im Kapitel \ref{sec:Valid_Batterie} beschrieben.


\paragraph{Audio Inputquelle auswählen \texttt{BSP\_SelectAudioIn}}

Es existieren 3 Audio Input-Quellen: Line-In, MIC, Ext-In.\\
Der TLV320 Codec hat eine interne Pfadbeschaltung um zwischen MIC und Line-In auszuwählen.
Mit dem externen Analog-Swich IC wird der Line-In zusätzlich aus Ext-In und Line-In gemultiplext.
Die Funktion \texttt{BSP\_SelectAudioIn} nimmt die Konfiguration des Audiopfades sowohl  beim Codec als auch beim Analog Multiplexer vor.
Durch einen einzigen Funktionsaufruf wird über eine GPIO Operation und der Verwendung der \texttt{tlv320aic}-Library der korrekte Audiopfad ausgewählt.

\begin{lstlisting}[style=Cuvision,caption={Audiopfad Input auf Line-In setzen.}]
BSP_SelectAudioIn(AUDIO_IN_LINE);
\end{lstlisting}



