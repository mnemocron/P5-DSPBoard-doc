\subsection{Ender 3 Pro}
\label{sec:Ender3Pro}


Der Ender 3 Pro der Firma Creality besitzt standardmässig bereits eine stabile Mechanik, welche in diesem Projekt grösstenteils übernommen wurde. Die wichtigsten Elemente der Hardware sind in der Tabelle \ref{tab:ender3prohardwarespecs} aufgeführt.

\begin{table}[h]
	\small
	\begin{center}
	\def\arraystretch{1.15} \tabcolsep=12pt
	\begin{tabular}{|l|l|}
		\hline
			\textbf{Modell} & Ender 3 Pro  \\ \hline
			\textbf{Druckvolumen} & 220 x 220 x 250 mm  \\ \hline
			\textbf{Speisung (Input)} & 200 V - 240 V / 3.4 A  \\ \hline
			\textbf{Speisung (Output)} & DC 24 V / 350 W  \\ \hline
			\textbf{Grösse} & 440 x 440 x 465 mm  \\ \hline
			\textbf{Gewicht} & 6.9 kg  \\ \hline
			\textbf{Material Rahmen} & Aluminium  \\ \hline
			\textbf{Antriebsart} & Zahnriemen, Spindel  \\ \hline
			\textbf{Schrittmotor} & Creality 3D 42-34 RepRap 42 mm  \\ \hline
			\textbf{Lüfter} & 24V / 100 mA  \\ \hline
			\textbf{Display / HMI} & LCD Matrixanzeige 128x64, Drehgeber  \\ \hline
			\textbf{Temperatursensoren} & NTC 100 k$\Omega$  \\ \hline
			\textbf{Endschalter} & Creality 3-Pin Mikroschalter  \\ \hline
		\end{tabular} 
	\end{center}
	\caption{Hardwarespezifikationen Ender 3 Pro}
	\label{tab:ender3prohardwarespecs}
\end{table}

%Alle mechanischen Komponenten sowie das Display mit Drehgeber werden vom bestehenden Ender 3 Pro übernommen. Einzig die Hauptplatine wird durch die selbst entwickelte Steuerungsplatine ersetzt. Zusätzlich werden die Endschalter der X- und Y-Achse weggelassen und durch die StallGuard-Funktion der TMC Treiber ersetzt.


In den folgenden zwei Unterkapiteln wird auf die mechanischen und die elektrischen Komponenten des Ender 3 Pro eingegangen.