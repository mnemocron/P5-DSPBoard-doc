\subsection{Grundlagen des 3D-Drucks}
\label{sec:Grundlagen3DDruck}
\todo[inline]{Fertig - OL - In diesem Kapitel werden die Grundlagen zum 3D-Druck gegeben (Was ist das Ziel des 3D-Drucks, Grober Aufbau eines 3D-Druckers (Achsen, Antrieb, Heizbett, Extruder), unterschiedliche Materialien, unterschiedliche Typen von 3D-Druckern und wozu sie verwendet werden), Was muss eine Steuerung grob machen (Schrittmotoren ansteuern, Heizungen regeln); Für die Bedienung auf das Manual verweisen. - A: MR}

Ein 3D-Drucker druckt, wie der Name schon sagt, im 3-dimensionalen Raum. Das heist es wird nicht einfach ein Blatt mit Tinte in X- und Y-Koordinaten bedruckt, sondern es wird ein Gegenstand (meisst aus Kunststoff, im Folgenden Filament genannt) erstellt. \\

Ein 3D-Drucker besteht im groben aus zwei Teilen. Der Elektronik und der Mechanik, wobei die Mechanik wiederum aus dem Rahmen, den Motoren, dem Druckbett und dem Druckkopf besteht. Der Druckkopf besteht wiederum aus dem Extruder, welcher das Filament bewegt, dem Hotend , welches das Filament erhitzt, und der Nozzle, eine Düse mit meist 0.4mm Öffnung, durch diese das meist 1.75mm dicke Filament gedrückt wird, und zwei Lüftern, welche das Hotend  und das austretende Filament kühlen. Im nachfolgendem Bild \ref{pic:Printer} ist ein 3D-Drucker dargestellt um dies aufzuzeigen:

\begin{figure}[h]
	\centering
	\includegraphics[scale=1.2]{Printer.eps}
	\caption{Beispiel eines 3D-Druckers. Abgebildet ist der in diesem Projekt verwendete Ender 3 Pro \cite{Ender3Pro}.}
	\label{pic:Printer}
\end{figure}

Der 3D-Drucker funktioniert im Grundsatz so, dass das Filament im Druckkopf vom Hotend  erhitzt und geschmolzen und anschliessend vom Extruder durch das Nozzle gedrückt wird. Ausserhalb des Nozzles wird das geschmolzene Filament wieder abgekühlt, wodurch es erstarrt. Der Druckkopf wird dabei so bewegt, dass aus dem erstarrten Filament das gewünschte Modell entsteht. In der Regel wird das Modell in Schichten gedruckt. Das heisst die Bewegung in Z-Richtung ist ausschliesslich positiv. Dieses vorgehen nennt man auch Fused Deposition Modeling, oder kurz FDM, da die verschiedenen Ebenen zusammen schmelzen. Dies ist auch in der nachfolgenden Grafik \ref{pic:Printhead} dargestellt:\\


\begin{figure}[h]
	\centering
	\includegraphics[width=0.9\linewidth]{3dprint.png}
	\caption{Aufbau eines Druckkopf eines 3D-Druckers im Betrieb.}
	\label{pic:Printhead}
\end{figure}

Beim in diesem Projekt verwendeten Ender 3 Pro wird der Druckkopf in X- und Z-Richtung bewegt, während sich das Druckbett nur in Y-Richtung bewegt. Für die Bewegung in X- und Y-Richtung werden Keilriemen verwendet, welche mittels Zahnräder an Schrittmotoren angebracht sind. Der Druckkopf wird mittels einer Spindel an einem Schrittmotor in Z-Richtung bewegt. Die Genauigkeit und Präzision dieser Bewegungen sind extrem wichtig, da sie direkt für die Druckqualität verantwortlich sind. Wie dies bewerkstelligt wird, wird im nächsten Kapitel \ref{sec:Microstepping} erklärt.\\
Bei vielen Druckern ist das Druckbett heizbar. Dies ist dazu hilfreich, damit das zu druckende Objekt besser am Druckbett haftet. Ausserdem verhindert es, dass sich zum Beispiel ABS während dem Druck durch das Abkühlen verzieht.\\
Die Aufgabe der Elektronik ist es, die Motoren der X-, Y-, und Z-Achse und des Extruders präzise zu steuern und die Temperatur des Hotend s und des Druckbetts zu regulieren. In der Regel wird der Elektronik eine G-Code Datei übergeben, in welcher die Parameter der Heizung und die Daten zu den Bewegungen enthalten sind. Diese G-Code Datei wird normalerweise am Computer mit Hilfe eines sogenannten Slicers generiert. Der Slicer konvertiert eine 3D-Datei in eine vom 3D-Drucker umsetzbare G-Code.

\paragraph{Filamente}
Die richtige Auswahl des verwendeten Filaments ist kritisch für den Erfolg eines Drucks. Im FDM Bereich werden unter anderem Folgende Materialien verwendet:
\begin{itemize}
    \item \textbf{PLA:} PLA ist sehr Einsteigerfreundlich. Es ist allgemein einfach zu Drucken, braucht nicht zwingend eine beheizte Druckfläche und ist vergleichsweise günstig. Es besteht aus Maisstärke und ist sehr hart und steif.
    \item \textbf{ABS:} ABS hat eine seht hohe Festigkeit, Zähigkeit und Steifigkeit. Jedoch wird zum Ducken eine beheitztes Druckbett und bei grösseren Objekten sogar ein Gehäuse um den Drucker benötigt, da sich das Material beim abkühlen zusammenzieht. Ausserdem entstehen beim Drucken Dämpfe, die gesundheitsschädigend sein können.
    \item \textbf{PETG:} PETG ist auch relativ einfach zu Drucken und hat sehr gute mechanische Eigenschaften und kann so gut als Alternative zu PLA verwendet werden. Jedoch wird auch hier ein beheitztes Druckbett benötigt \cite{Filaments}.
\end{itemize}

Dies ist nur eine kleine Auswahl der verwendeten Filamente und soll nur die häufigsten/bekanntesten repräsentieren.


\paragraph{Limitationen des 3D-Drucks}
Eine Einschränkung beim 3D-Druck ist das Beschränkte Druckvolumen. Der Ender 3 Pro zum Beispiel hat ein Druckvolumen von $220\times 220 \times 250$mm \cite{Ender3Pro}. Das heisst das zu Druckende Objekt muss in jeder Dimension innerhalb dieses Bereiches liegen. Andernfalls ist es für den Drucker nicht möglich das Objekt zu drucken und es muss allenfalls in mehreren Teilen gedruckt und später zusammengefügt werden.\\
Eine weitere Einschränkung ist es, dass es im FDM Verfahren praktisch nicht möglich ist, steile Überhänge zu drucken. Dies liegt daran, dass das bei der Nozzle austretende, geschmolzene Filament darauf angewiesen ist, auf anderem Material zu liegen zu kommen. Deshalb ist es nötig bei steilen Überhängen eine weitere Stützstruktur mitzudrucken, welche nach erfolgreichen Druck wieder entfernt werden kann. Das erstellen dieser Stützstrukturen übernimmt meisst das Slicer Programm. Das entfernen dieser Stützstrukturen kann jedoch recht aufwändig sein und hinterlässt Spuren am finalen Objekt.

\paragraph{Weitere Druckverfahren}
Der Vollständigkeit halber sind nachfolgend weitere populäre 3D Druckverfahren aufgelistet:
\begin{itemize}
    \item \textbf{3DP:} Beim 3D-Druck mit Pulver fliesst Leim aus dem Druckkopf, welches auf ein Gipsartiges Pulver aufgetragen wird und dies verfestigt. Dies führt zu einem steinartigen Aussehen.
    \item \textbf{SLS:} Beim Selective Laser Sintering wird ein Metallpulver von einem Laser geschmolzen und so zusammen verschmolzen. Dies ergibt äusserst stabile und trotzem filigrane Objekte.
    \item \textbf{Stereolithografie:} Bei der Stereolithographie wird Epoxidharz von einem Laser bestrahlt und so verfestigt \cite{Druckverfahren}.
\end{itemize}