\subsubsection{ESP32}
\label{sec:SchemaESP32}

Hier wird das Teilschema \texttt{appendix\_ESP32.sch} beschrieben. Auf dieser Ebene sind die Bauteile mit Nummern ab 300 beschriftet.\\

Um mit der Steuerung per WiFi kommunizieren zu können, wird ein ESP32-WROOM von Espressif Systems verbaut \cite{ESP32Wroom}. Dies ist ein populäres WiFi Modul mit eigenem Controller und integrierter Antenne, was den Vorteil hat, dass keine impedanzkritischen HF-Schaltungen realisiert werden müssen. Weiter existiert bereits eine Applikation für das ESP32 mit der Marlin aus einem Webbrowser gesteuert werden kann, siehe Kapite \ref{sec:ESP3D}. Dabei werden Anweisungen vom Benutzer auf dem integrierten Webserver entgegengenommen und über eine Serielle Schnittstelle an den STM32 und Marlin gesendet. 

\paragraph{UART Schnittstellen}
Der ESP32 hat drei interne UART Schnittstellen. \texttt{UART0} ist für die Programmierung reserviert. 
\texttt{UART1} wird für den auf dem Modul vorhandenen Flash-Speicher verwendet und sollte nicht noch zusätzlich belegt werden.
Die noch unbelegte \texttt{UART2} Schnittstelle wird von der Firmware ESP3D für die Kommunikation mit der Firmware Marlin verwendet. Somit ist sie über die Pins \texttt{GPIO16} (RX) und \texttt{GPIO17} (TX) mit dem STM32F103 verbunden.
Vorteilhaft an der Pinbelegung des ESP32 Mikrocontrollers ist, dass die UART Pins einem beliebigen GPIO-Pin zugeordnet werden können.

\paragraph{Bootpins und Reset}
Der ESP32 wird in den Bootloader-Modus versetzt, wenn \texttt{GPIO0} während einem Reset auf \texttt{LOW} gehalten wird \cite{ESP32BootMode}.
Wenn sich der ESP32 im Bootloader-Modus befindet, kann die Firmware über die UART Schnittstelle in dessen Speicher geladen werden.
Für die Ablaufsteuerung beim Programmieren wird eine gängige Schaltung mit zwei Bipolartransistoren (siehe Abbildung \ref{pic:ESP32_Schema}) verwendet. Diese ist in diversen Open-Source-Hardware Projekten wie beispielsweise dem Adafruit HUZZAH32 zu finden \cite{AdafruitHuzzah32}.\\


\paragraph{ESP3D Einstellungen zurücksetzen}
Die ESP3D Firmware erlaubt es, durch einen extern angebrachten Taster, die Wi-Fi Einstellungen vom Controller zurückzusetzen. Der Button ist an folgendem Pin angeschlossen:

\begin{table}[h]
\small
	\begin{center}
	\def\arraystretch{1.3} \tabcolsep=12pt
		\begin{tabular}{|l|l|l|}
			\hline
			 \textbf{Pin (ESP32)} & \textbf{Pin (ESP32-WROOM-32 Modul)} & \textbf{GPIO} \\ \hline
			 \texttt{14} & \texttt{10} & \texttt{GPIO25} \\ \hline
		\end{tabular} 
	\end{center}
%	\caption{•}
%	\label{tab:•}
\end{table}



\begin{figure}[H]
	\centering
	\includegraphics[width=0.85\linewidth]{ESp_Schema.pdf}
	\caption{Schemaausschnitt vom ESP32 Modul und der Beschaltung}
	\label{pic:ESP32_Schema}
\end{figure}
