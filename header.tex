%%---Main Packages-----------------------------------------------------------------------
\usepackage[english, ngerman]{babel}	%Mul­tilin­gual sup­port for LaTeX
\usepackage[T1]{fontenc}				%Stan­dard pack­age for se­lect­ing font en­cod­ings
\usepackage[utf8]{inputenc}				%Ac­cept dif­fer­ent in­put en­cod­ings
\usepackage{lmodern}                    %The newer Font-Set
\usepackage{textcomp}					%LaTeX sup­port for the Text Com­pan­ion fonts
\usepackage{graphicx} 					%En­hanced sup­port for graph­ics
\usepackage{float}						%Im­proved in­ter­face for float­ing ob­jects
\usepackage{ifdraft}                    %Let you check if the doc is in draft mode

%%---Useful Packages---------------------------------------------------------------------
\usepackage[pdftex,dvipsnames]{xcolor}  %Driver-in­de­pen­dent color ex­ten­sions for LaTeX
\usepackage{csquotes}                   %Simpler quoting with \enquote{}
\usepackage{siunitx} 					%A com­pre­hen­sive (SI) units pack­age
\usepackage{listings}					%Type­set source code list­ings us­ing LaTeX
\usepackage[bottom]{footmisc}			%A range of foot­note op­tions
\usepackage{footnote}					%Im­prove on LaTeX's foot­note han­dling
\usepackage{verbatim}					%Reim­ple­men­ta­tion of and ex­ten­sions to LaTeX ver­ba­tim
%\usepackage[textsize=footnotesize,disable]{todonotes} %Mark­ing things to do in a LaTeX doc­u­ment
\usepackage[colorinlistoftodos,prependcaption,textsize=tiny]{todonotes} %Mark­ing things to do in a LaTeX doc­u­ment
\usepackage{titling}					%Control over the typesetting of the \maketitle command
\usepackage{pbox}


%%---Tikz Packages-----------------------------------------------------------------------
\usepackage{standalone}
\usepackage{tikz}
\usepackage{circuitikz}
\usetikzlibrary{arrows}
\usetikzlibrary{calc}
\usetikzlibrary{intersections}

%%---Math Packages-----------------------------------------------------------------------
\usepackage{amsmath}					%AMS math­e­mat­i­cal fa­cil­i­ties for LaTeX
\usepackage{amssymb}					%Type­set­ting symbols (AMS style)
%\usepackage{array}						%Ex­tend­ing the ar­ray and tab­u­lar en­vi­ron­ments
%\usepackage{amsthm}					%Type­set­ting the­o­rems (AMS style)

%%---Table Packages----------------------------------------------------------------------
\usepackage{tabularx}					%Tab­u­lars with ad­justable-width columns
\usepackage{longtable}
\usepackage{multirow}					%Create tab­u­lar cells span­ning mul­ti­ple rows
\usepackage{multicol}					%In­ter­mix sin­gle and mul­ti­ple columns

%%---PDF / Figure Packages---------------------------------------------------------------
\usepackage{pdfpages}					%In­clude PDF doc­u­ments in LaTeX
\usepackage{pdflscape}					%Make land­scape pages dis­play as land­scape
\usepackage{subfig}					    %Fig­ures di­vided into sub­fig­ures

%%---Other Packages----------------------------------------------------------------------
%\usepackage{xargs}                     %De­fine com­mands with many op­tional ar­gu­ments


%%---Bibliography------------------------------------------------------------------------
\usepackage[style=ieee,urldate=comp,backend=biber]{biblatex}
\addbibresource{literature/bibliography.bib}

%%---Main Settings-----------------------------------------------------------------------
\graphicspath{{./graphics/}}			%Defines the graphicspath
\geometry{twoside=false}				    %twoside=false disables the "bookstyle"
\setlength{\marginparwidth}{2cm}
\overfullrule=5em						%Creates a black rule if text goes over the margins => debugging


\hypersetup{
	colorlinks=true,
	linkcolor=black,
	filecolor=blue,      
	urlcolor=blue,
	citecolor=black
}

%%---User Definitions--------------------------------------------------------------------
%%Tabel-Definitions: (requires \usepackage{tabularx})
\newcolumntype{L}[1]{>{\raggedright\arraybackslash}p{#1}}    %column-width and alignment
\newcolumntype{C}[1]{>{\centering\arraybackslash}p{#1}}
\newcolumntype{R}[1]{>{\raggedleft\arraybackslash}p{#1}}

%%---Optional Package Settings-----------------------------------------------------------
%Listings-Settings: (requires \usepackage{listings}) => Example with Matlab Code
\lstset{ 
	backgroundcolor=\color{white},   % choose the background color; you must add \usepackage{color} or \usepackage{xcolor}; should come as last argument
	basicstyle=\footnotesize\ttfamily,   % the size of the fonts that are used for the code
	breakatwhitespace=false,         % sets if automatic breaks should only happen at whitespace
	breaklines=false,                 % sets automatic line breaking
	captionpos=b,                    % sets the caption-position to bottom
	escapeinside={\%*}{*)},          % if you want to add LaTeX within your code
	firstnumber=1,                   % start line enumeration with line 1
	frame=single,	                   % adds a frame around the code
	keepspaces=true,                 % keeps spaces in text, useful for keeping indentation of code (possibly needs columns=flexible)
	keywordstyle=\color{blue},       % keyword style
	language=c,                 % the language of the code
	numbers=left,                    % where to put the line-numbers; possible values are (none, left, right)
	numbersep=5pt,                   % how far the line-numbers are from the code
	numberstyle=\color{mygray}, % the style that is used for the line-numbers
	rulecolor=\color{black},         % if not set, the frame-color may be changed on line-breaks within not-black text (e.g. comments (green here))
	showspaces=false,                % show spaces everywhere adding particular underscores; it overrides 'showstringspaces'
	showstringspaces=false,          % underline spaces within strings only
	showtabs=false,                  % show tabs within strings adding particular underscores
	tabsize=2,	                   % sets default tabsize to 2 spaces
	title=\lstname                   % show the filename of files included with \lstinputlisting; also try caption instead of title
}

\lstset{language=Matlab,%
	captionpos=b,                    % sets the caption-position to bottom
    basicstyle=\footnotesize\ttfamily,
    breaklines=false,%
    morekeywords={switch, case, otherwise},
    keywordstyle=\color{Blue},%
    tabsize=2,
    %morekeywords=[2]{1}, keywordstyle=[2]{\color{black}},
    identifierstyle=\color{Black},%
    stringstyle=\color{Purple},
    commentstyle=\color{Green},%
    showstringspaces=false,%without this there will be a symbol in the places where there is a space
    numbers=left,%
    numberstyle={\tiny \color{black}},% size of the numbers
    numbersep=9pt, % this defines how far the numbers are from the text
    %emph=[1]{word1, word2,...},emphstyle=[1]\color{red}
}		

\lstdefinestyle{Cuvision}{
	language=C,
	captionpos=b,                    % sets the caption-position to bottom
	morekeywords={int8_t, uint8_t, int16_t, uint16_t, int32_t, uint32_t, float32_t}
	keywordstyle=\color{Blue},
	identifierstyle=\color{Black},%
	stringstyle=\color{Purple},
	commentstyle=\color{Green},%
	frame=single,	                   % adds a frame around the code
	keepspaces=true,
}

% Hurenkinder und Schusterjungen verhindern ( kein scherz Google es)
\clubpenalty10000
\widowpenalty10000
\displaywidowpenalty=10000	